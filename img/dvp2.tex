\documentclass[tikz]{standalone}
\usepackage{tikz}
\usetikzlibrary{arrows}
\usepackage[english,russian]{babel}
\usepackage[T2A,T1]{fontenc}
\usepackage[utf8]{inputenc}
\usepackage{verbatim}
\makeatletter
\tikzoption{canvas is plane}[]{\@setOxy#1}
\def\@setOxy O(#1,#2,#3)x(#4,#5,#6)y(#7,#8,#9)%
  {\def\tikz@plane@origin{\pgfpointxyz{#1}{#2}{#3}}%
   \def\tikz@plane@x{\pgfpointxyz{#4}{#5}{#6}}%
   \def\tikz@plane@y{\pgfpointxyz{#7}{#8}{#9}}%
   \tikz@canvas@is@plane
  }
\makeatother  

\usetikzlibrary{patterns,angles,quotes}
\usetikzlibrary{calc, 3d, backgrounds, positioning}
% \pgfplotsset{compat=newest}

\begin{document}
\begin{tikzpicture}[scale=1.25]
    \xdef\a{2}
    \xdef\b{1}
    \xdef\angle{45}
    \pgfmathsetmacro{\e}{sqrt(1-\b^2/\a^2)}%
    \pgfmathsetmacro{\p}{\b^2/\a)}%
    \pgfmathsetmacro{\c}{sqrt(\a^2-\b^2)}%


    % \draw[dashed, black!40] (1.31,0) -- ++(0,0.76);
% 
    % \draw[black, thick] (1.31,-.1) node[below,xshift=0.35em] {$n_{ellipse}$} -- ++(0,0.2);

    % \draw[dashed, black!40] (30:1) -- ++(-{cos(30)},0);
    % \draw[thin,samples=200,domain=0:360,smooth,variable=\x,magenta] plot ({\a*sin(\x)},{\b*cos(\x)});
    \draw[thin,samples=200,domain=0:360,smooth,variable=\x,magenta] plot ({\b*sin(\x)},{\b*cos(\x)});

    \draw[thin,samples=200,domain=0:360,smooth, xshift=-{\c}cm,variable=\phi,magenta]%
        plot ({\phi}:{\p/(1-\e*cos(\phi))});

    \begin{scope}[rotate around={30:(0,0)}]
        \draw[blue,->, thick] (0,0) -- (1.5,0) node[above, xshift=1em] {$\vec{E}$};
        \draw[blue, thick] (1,-0.1)  node[left, xshift=-0.3em, yshift=0.3em] {$n_0$} -- ++(0,0.2);
        \draw[blue, thick] (1.5,-0.1)  node[left, xshift=-0.3em, yshift=0.3em] {$n$} -- ++(0,0.2);

    \end{scope}

    % \draw
    \draw[black,->, thin] (0,0) -- (2.5,0);% node [right] {$+y$};
    \draw[black,->, thin] (0,0) -- (0,1.5);% node [above] {$+x$};

    % \draw[blue,->,thick] (0,0) -- (30:1.50) 
    % node [above, black] {$n,$}
    % node [above right] {$\vec{E}$};

    \draw[blue,->,thick] (0,0) -- ++(1.31,0);
    \draw[blue,->,thick] (0,0) -- ($(30:1.5)+(-{1.5*cos(30)},0)$);
% 

    \draw[black,thick] (2,-.1) node[right,yshift=-0.5em] {$n_1$} -- ++(0,0.2);
    \draw[black, thick] (-.1,1) node[above, xshift=-0.7em] {$n_0$} -- ++(0.2,0);    
    % \draw[->, thick] (0,0) -- ++({\a*sin(45)},{\b*cos(45)}) coordinate (o) node [above, xshift=1em] {$\vec{E}$};

    % \draw[->, thick] (0,0) -- ++(45:2) coordinate (o) node [above, xshift=1em] {$\vec{E}$};    

    % \draw[->, thick] (0,0) -- ++({\a*sin(0)},{cos(45)}) node [left, xshift=0em] {$\vec{E}_1$};

    % \draw[dashed]
    %             ({\a*sin(0)},{cos(45)}) --
    %             (o);

    % \draw[dashed]
    %             ({\a*sin(45)},{cos(90)}) --
    %             (o);                

    % \draw[->, thick] (0,0) -- ++({\a*sin(45)},{cos(90)}) node [below, xshift=-0.2em] {$\vec{E}_2$};
    % \draw (2,0) -- ++(0,2);
    % \foreach \i in {0.9,0.95,...,1.1}{
    %     \draw[magenta,->] (-1,\i) -- ++ (4,0);
    % };
    % \foreach \i in {0.9,0.95,...,1.1}{
    %     \draw[magenta] (0,\i) -- ++ (2,\i+0.5cm);
    % };
    % \foreach \i in {0.9,0.95,...,1.1}{
    %     \draw[magenta,->] (2,{\i+0.535})-- ++(1,0);
    % };

    % % \draw[magenta,->]
    % % \coordinate (b)  at (1,1.1);
    % % \coordinate (a)  at (2,1.1);
    % % \coordinate (c)  at (2,2.1);
    % % \draw (c) pic["$\alpha$", draw=orange, <->, angle eccentricity=1.2, angle radius=1cm]
    % % {angle=a--b--c};
\end{tikzpicture}
% В анизатропной среде можно построить эллипсоид коэффециента преломления.
% Волны $E_1$ и $E_2$ линейно поляризованы и их поляризации $\perp$ друг другу, а их скорости 
\end{document}