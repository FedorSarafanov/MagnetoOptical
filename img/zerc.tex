\documentclass[tikz]{standalone}
\usepackage{tikz}
\usetikzlibrary{arrows}
\usepackage[english,russian]{babel}
\usepackage[T2A,T1]{fontenc}
\usepackage[utf8]{inputenc}
\usepackage{pgfplots}
\usepackage{verbatim}
\makeatletter
\tikzoption{canvas is plane}[]{\@setOxy#1}
\def\@setOxy O(#1,#2,#3)x(#4,#5,#6)y(#7,#8,#9)%
  {\def\tikz@plane@origin{\pgfpointxyz{#1}{#2}{#3}}%
   \def\tikz@plane@x{\pgfpointxyz{#4}{#5}{#6}}%
   \def\tikz@plane@y{\pgfpointxyz{#7}{#8}{#9}}%
   \tikz@canvas@is@plane
  }
\makeatother  

\usetikzlibrary{
    patterns,
    angles,
    quotes,
    calc, 
    3d, 
    backgrounds, 
    positioning
}
\begin{document}
\begin{tikzpicture}
    %источник
    \begin{scope}[xshift=-0.5cm]
        \draw (0,0.5) rectangle node[above, yshift=1em] {1} ++(0.5,0.5);
    \end{scope}
    %поляризатор
    \begin{scope}[xshift=0.5cm]
        \draw (0,0.5) rectangle node[above, yshift=1em] {2} ++(0.5,0.5);
        \draw (0,0.5) -- ++ (0.5,0.5);
    \end{scope}
    \draw[magenta, line width=1pt] (0,0.80) -- ++ (0.5,0);  
    \draw[magenta, line width=1pt] (1,0.80) -- ++ (5.5,0);  
    \draw[blue, line width=1pt] (1,0.70) -- ++ (5.5,0);  
    %поляризация туда
    \begin{scope}[xshift=0cm]
        \xdef\angle{135}
        \draw [-,magenta, line width=1pt] (2,0.75) -- ++({180+\angle}:0.3);
        \draw [->,magenta, line width=1pt] (2,0.75) -- ++({\angle}:0.3);
    \end{scope}  
    %поляризация обратно
   \begin{scope}[xshift=0cm]
        % \draw[fill=white, dashed] (2,0.75) circle (0.25);
        \xdef\angle{45}
        \draw [-,blue, line width=1pt] (2,0.75) -- ++({180+\angle}:0.3);
        \draw [->,blue, line width=1pt] (2,0.75) -- ++({\angle}:0.3);
    \end{scope}      
    %поляризация до отражения
    \begin{scope}[xshift=3cm]
        % \draw[fill=white, dashed] (2,0.75) circle (0.25);
        \xdef\angle{90}
        \draw [-,magenta, line width=1pt] (2,0.75) -- ++({180+\angle}:0.3);
        \draw [->,magenta, line width=1pt] (2,0.75) -- ++({\angle}:0.3);
    \end{scope}   
    %поляризация после отражения
    \begin{scope}[xshift=3.75cm]
        % \draw[fill=white, dashed] (2,0.75) circle (0.25);
        \xdef\angle{90}
        \draw [-,blue, line width=1pt] (2,0.75) -- ++({180+\angle}:0.3);
        \draw [->,blue, line width=1pt] (2,0.75) -- ++({\angle}:0.3);
    \end{scope}       
    %фильтр фарадея
    \begin{scope}[xshift=3cm]
      \draw[fill=white] (0,0.5) rectangle node[above, yshift=1em] {3} ++(1,0.5);
    \end{scope}  
    %зеркало
    \draw[blue,line width=.5pt] (6.5,0) -- ++(0,1.5);
    \draw[pattern=north west lines, pattern color=blue, draw=none] (6.5,0) rectangle ++(0.25,1.5); 
    %филл
    \draw[draw=none] (0,0) -- ++(6.75,0);
\end{tikzpicture} 
\end{document}