
\begin{frame}

\frametitle{Поворот поляризации}
\begin{enumerate} 
	\item
	Для простоты предположим, что начальная фаза волны равна нулю.
	\begin{gather*}
		\begin{cases} 
			E_x = A\cos(\xi)\cos\left(-kz+\omega t\right) \\
			E_y = A\sin(\xi)\cos\left(-kz+\omega t\right)
			 % \\+ \phi_2\right
			% E_z = 0
		\end{cases}
		% \quad\Rightarrow\quad
		% \frac{E_x^2}{E_1^2}-\frac{2E_xE_y}{E_1E_2}\cos\delta+\frac{E_y^2}{E_2^2}=\sin^2\delta
	\end{gather*}
	\item
	Предположим, что поворот поляризации линейно зависит от $z$:
	\begin{equation*}
		\xi=-\alpha z
	\end{equation*}
	% \begin{equation*}
	% 	\cos x \cos y = \frac12 \left[
	% 		\cos (x-y)+
	% 		\cos (x+y)
	% 	\right]
	% \end{equation*}
	% \begin{equation*}
	% 	\sin x \cos y = \frac12 \left[
	% 		\sin (x+y)+
	% 		\sin (x-y)
	% 	\right]
	% \end{equation*}
	\begin{gather*}
		\begin{cases} 
			E_x = \frac{A}{2}\left[
				\cos\left(
					\xi+kz-\omega t
				\right)+
				\cos\left(
					\xi-kz+\omega t
					\right)
			.\right] \\
			E_y = \frac{A}{2}\left[
				\sin\left(
					\xi-kz+\omega t
				\right)+
				\sin\left(
					\xi+kz-\omega t
					\right)
			\right]
			 % \\+ \phi_2\right
			% E_z = 0
		\end{cases}
		% \quad\Rightarrow\quad
		% \frac{E_x^2}{E_1^2}-\frac{2E_xE_y}{E_1E_2}\cos\delta+\frac{E_y^2}{E_2^2}=\sin^2\delta
	\end{gather*}
	\item%
	\begin{gather*}
		\begin{cases} 
			E_x = \frac{A}{2}\left[
				\cos\left(
					-z(k-\alpha)+\omega t
				\right)+
				\cos\left(
					-z(k+\alpha)+\omega t
					\right)
			\right] \\
			E_y = \frac{A}{2}\left[
				\cos\left(
					-z(k-\alpha)+\omega t+\frac{\pi}{2}
				\right)+
				\cos\left(
					-z(k+\alpha)+\omega t-\frac{\pi}{2}
					\right)
			\right]
			 % \\+ \phi_2\right
			% E_z = 0
		\end{cases}
		% \quad\Rightarrow\quad
		% \frac{E_x^2}{E_1^2}-\frac{2E_xE_y}{E_1E_2}\cos\delta+\frac{E_y^2}{E_2^2}=\sin^2\delta
	\end{gather*}
\end{enumerate}
\end{frame}


\begin{frame}
	\begin{enumerate} 
	\setcounter{enumi}{3}
	\item
	Представим через суперпозицию, где $k^R=k-\alpha$, $k^L=k+\alpha$:
	\begin{gather*}
		\begin{cases}
		\begin{cases} 
			E_x^R = \frac{A}{2}
				\cos\left(
					\omega t - k^Rz
					\right)
			\\
			E_y^R = \frac{A}{2}
				\cos\left(
					\omega t - k^Rz +\frac{\pi}{2}
					\right)
		\end{cases}\vspace{0.5em}\\
		\begin{cases} 
			E_x^L = \frac{A}{2}
				\cos\left(
					\omega t - k^Lz
					\right)
			\\
			E_y^L = \frac{A}{2}
				\cos\left(
					\omega t - k^Lz -\frac{\pi}{2}
					\right)
		\end{cases}			
		\end{cases}
		% \quad\Rightarrow\quad
		% \frac{E_x^2}{E_1^2}-\frac{2E_xE_y}{E_1E_2}\cos\delta+\frac{E_y^2}{E_2^2}=\sin^2\delta
	\end{gather*}
	\begin{equation*}
		\omega=2\pi\nu,\quad
		\lambda=\frac{2\pi}{k},\quad\Rightarrow\quad
		v=\lambda\nu=\frac{\omega}{k}
	\end{equation*}
	\item
	Тогда выразим скорости и показатели преломления этих волн:
\begin{gather*}
	v_L=\frac{\omega}{k-\alpha},
	\quad
	v_R=\frac{\omega}{k+\alpha},
	\quad
	n_L=\frac{c}{v_L},
	\quad
	n_R=\frac{c}{v_L}
\end{gather*}
откуда 
\begin{equation*}
	n_L-n_R=\frac{2c}{\omega}\alpha
\end{equation*}
\item
\begin{equation*}
	\alpha=\frac{\omega}{2c}(n_L-n_R)
\end{equation*}
\end{enumerate} 
\end{frame}