\documentclass[tikz]{standalone}

\usepackage[english,russian]{babel}
\usepackage[T2A,T1]{fontenc}
\usepackage[utf8x]{inputenc}
\usepackage
    {
        tikz,
        pgfplots,
        verbatim,
        tikz-3dplot
    }
\usetikzlibrary
    {
        arrows,
        patterns,
        angles,
        quotes,
        calc, 
        3d, 
        backgrounds, 
        positioning,
        babel
    }
\tikzset{
    axis/.style={dashed,black!80},
    light axis/.style={dashed,black!50},
    vec line/.style={solid, thick, blue},
    vec/.style={solid, thick, blue,-latex},
}

\begin{document}

\begin{tikzpicture}[
		x={(1cm*2,0cm)},
		y={(0,1cm*2)},
		z={(0,-0.2cm*2)},
		rotate=90+180,
	]
	
	%Ось координат
	\draw[dashed] (0,-2, 0) -- (0,2,0); 
	
	%Луч
	\draw[color=orange, thick] (0,0,0)
	-- (xyz spherical cs:radius=8,longitude=-45,latitude=-85) coordinate (*);
	
	\begin{scope}[
			canvas is xz plane at y=0
		]
		\draw[axis] (0,0) circle (1);
		\draw[vec line] (0,0) -- (180:1) coordinate (A);
		\draw[vec] (*) -- ++(180:0.5) coordinate (np);
	\end{scope}
	
	
	\begin{scope}[ canvas is xy plane at z=0]
		\draw (0,0) ellipse (1 and 1.5);
	\end{scope}
	
	\begin{scope}[
			canvas is xy plane at z=0,
			rotate around y=-65,
		]
		\draw[axis] (0,0) ellipse (1 and 1.5);
		\draw[light axis] (-1,0) -- (1,0);
		\draw[vec line] (0,0) -- (45:1.18)  coordinate (B);
		\draw[vec line] (90:1.5)  coordinate (o);
		\draw[vec] (*) -- ++(45:0.5) coordinate (ne);
	\end{scope}
	
	\draw[fill=magenta] (A) circle (2pt) node[above] {$n_0$};
	\draw[fill=magenta] (B) circle (2pt) node[above] {$n_e$};
	% \draw[fill=magenta] (o) circle (2pt) node[right, yshift=-0.6em] {$n_1$};
	\draw[] (np) node[above] {$\vec{E}_\perp$};
	\draw[] (ne) node[right] {$\vec{E}_\parallel$};
	
\end{tikzpicture}
\end{document}