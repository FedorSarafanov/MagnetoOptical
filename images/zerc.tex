\documentclass[tikz]{standalone}

\usepackage[english,russian]{babel}
\usepackage[T2A,T1]{fontenc}
\usepackage[utf8x]{inputenc}
\usepackage
    {
        tikz,
        pgfplots,
        verbatim,
        tikz-3dplot
    }
\usetikzlibrary
    {
        arrows,
        patterns,
        angles,
        quotes,
        calc, 
        3d, 
        backgrounds, 
        positioning,
        babel
    }

\usepackage{ifthen}
\usepackage[outline]{contour}

\newcommand{\polarize}[4][magenta]{%
    \begin{scope}[xshift={#2 cm}, yshift=0.75cm]
        \draw[dashed,->] (0,0.75) -- ++(90:0.75) node[above] {$y$};
        \draw[dashed,->] (0,0.75) -- ++(0:0.75) node[right] {$x$};
        \draw[dashed] (0,0.75) -- ++(#3:0.75);
        \draw (0,0.75) circle (0.3cm);
        \xdef\angle{#3}
        \draw [-,#1, line width=1pt] (0,0.75) -- ++({180+\angle}:0.3);
        \draw [->,#1, line width=1pt] (0,0.75) -- ++({\angle}:0.3);

        \ifthenelse{\angle=90}{}
        {
            \draw[dashed] (0,0.75) coordinate (a) node[right] {}
            -- (0,0.75) coordinate (b) node[left] {}
            -- (0,0.75)++(#3:1) coordinate (c) node[above right] {}
            pic["$#4$", draw=black, <-, solid, angle eccentricity=1.5, angle radius=0.7cm]
                {angle=c--b--a};
        }

    \end{scope}
}

\contourlength{1.2pt}

\newcommand{\polvec}[3][magenta]{%
    \begin{scope}[xshift={#2 cm}, yshift=0cm]
        \xdef\angle{#3}

        \ifthenelse{\angle=90}{
        \draw[#1] (0,0.75) node [solid] {\contour{white}{$\bigotimes$}}; 

    % \draw [fill=gray!10!white](0,0) rectangle (4.5,1) [step=0.1cm] (0,0) grid (4.5,1);
    % \node at (0.75,0.5) {\contour{white}{\Large Text!}};
        % \draw [-,#1, line width=1pt] (0,0.75) -- ++({180+90}:{0.25*cos(\angle)});
        % \draw [->,#1, line width=1pt] (0,0.75) -- ++({90}:{0.25*cos(\angle)});
        }
        {
        \draw [-,#1, line width=1pt] (0,0.75) -- ++({180+90}:{0.25*cos(\angle)});
        \draw [->,#1, line width=1pt] (0,0.75) -- ++({90}:{0.25*cos(\angle)});
        }
    \end{scope}
}
\begin{document}
\begin{tikzpicture}[scale=1.4]


    \draw[yshift=0.75cm,->, dashed] (-1.5,0) -- ++(0.6,0) node[right] {$z$};
    \draw[yshift=0.75cm,->, dashed] (-1.5,0) -- ++(0,0.6) node[above] {$y$};
    \draw[yshift=0.75cm,->, dashed] (-1.5,0) node [solid, fill=white] {$\bigotimes$} node[below, yshift=-0.5em] {$x$};

    %источник
    \begin{scope}[xshift=-0.5cm]
        \draw (0,0.5) rectangle node[above, yshift=1em] {1} ++(0.5,0.5);
    \end{scope}
    %поляризатор
    \begin{scope}[xshift=0.5cm]
        \draw (0,0.5) rectangle node[above, yshift=1em] {2} ++(0.5,0.5);
        \draw (0,0.5) -- ++ (0.5,0.5);
    \end{scope}
    \draw[magenta, line width=1pt] (0,0.80) -- ++ (0.5,0);  
    \draw[magenta, line width=1pt] (1,0.80) -- ++ (5.5,0);  
    \draw[blue, line width=1pt] (1,0.70) -- ++ (5.5,0);  
    %поляризация туда
    % \begin{scope}[xshift=0cm]
    %     \xdef\angle{135}
    %     \draw [-,magenta, line width=1pt] (2,0.75) -- ++({180+\angle}:0.3);
    %     \draw [->,magenta, line width=1pt] (2,0.75) -- ++({\angle}:0.3);
    % \end{scope}  
    %поляризация обратно
    % \polarize[magenta]{2}{90}{\frac{\pi}{2}}
    \polarize[magenta]{2}{90}{}
    \polarize[blue]{2}{0}{\frac{\pi}{2}}

    \polvec[magenta]{2}{0}
    \polvec[blue]{2.3}{90}

    \polarize[magenta]{5}{45}{}
    \polarize[blue]{5}{45}{\frac{\pi}{4}}
    \polarize[magenta,dashed]{5}{45}{}

    \polvec[magenta]{5}{45}
    \polvec[blue]{5.3}{45}
   % \begin{scope}[xshift=0cm]
   %      % \draw[fill=white, dashed] (2,0.75) circle (0.25);
   %      \xdef\angle{45}
   %      \draw [-,blue, line width=1pt] (2,0.75) -- ++({180+\angle}:0.3);
   %      \draw [->,blue, line width=1pt] (2,0.75) -- ++({\angle}:0.3);
   
   % \draw[dashed] (2,0.75) -- ++ (135:0.75);
   %  \draw[dashed] (1.5,0.75) coordinate (a) node[right] {}
   %  -- (2,0.75) coordinate (b) node[left] {}
   %  -- (2,0.75)++(135:1) coordinate (c) node[above right] {}
   %  pic["$\varphi_0$", draw=black, <-, solid, angle eccentricity=1.7, angle radius=0.4cm]
   %  {angle=c--b--a};

   % \draw[dashed] (2,0.75) -- ++ (45:0.75);
   %  \draw[dashed] (1.5,0.75) coordinate (a) node[right] {}
   %  -- (2,0.75) coordinate (b) node[left] {}
   %  -- (2,0.75)++(45:1) coordinate (c) node[above right] {}
   %  pic["$\varphi_0+\frac{\pi}{2}$", draw=black, <-, solid, angle eccentricity=1.6, angle radius=0.47cm]
   %  {angle=c--b--a};
   %  \end{scope}      
    % %поляризация до отражения
    % \begin{scope}[xshift=3cm]
    %     % \draw[fill=white, dashed] (2,0.75) circle (0.25);
    %     \xdef\angle{90}
    %     \draw [-,magenta, line width=1pt] (2,0.75) -- ++({180+\angle}:0.3);
    %     \draw [->,magenta, line width=1pt] (2,0.75) -- ++({\angle}:0.3);
    % \end{scope}   
    % %поляризация после отражения
    % \begin{scope}[xshift=3.15cm]
    %     % \draw[fill=white, dashed] (2,0.75) circle (0.25);
    %     \xdef\angle{90}
    %     \draw [-,blue, line width=1pt] (2,0.75) -- ++({180+\angle}:0.3);
    %     \draw [->,blue, line width=1pt] (2,0.75) -- ++({\angle}:0.3);

    % \draw[dashed] (2,1.75) coordinate (a) node[right] {}
    % -- (2,0.75) coordinate (b) node[left] {}
    % -- (2,0.75)++(180:1) coordinate (c) node[above right] {}
    % pic["$\varphi_0+\frac{\pi}{4}$", draw=black, <-, solid, angle eccentricity=2.1, angle radius=0.4cm]
    % {angle=a--b--c};
    % \end{scope}       
    %фильтр фарадея
    % \begin{scope}[xshift=3cm]
    %   \draw[fill=white] (0,0.5) rectangle node[above, yshift=1em] {3} ++(1,0.5);
    % \end{scope}  
    \begin{scope}[xshift=3cm, yshift=0.75cm]
      \draw[line width=3pt] (0,-0.25) -- ++ (1,0);
      \draw[line width=3pt] (0,0.25) -- node[above, yshift=0.2em] {3}  ++ (1,0);
      % \draw[magenta, line width=1pt] (0,0) -- (1,0);
      \draw[fill=white] (0.1,-0.1) rectangle ++ (0.8,0.2);
      % \draw[magenta!30, line width=1pt] (0.1,1pt) --++ (0.8,0);
    \end{scope}  
    %зеркало
    \draw[blue,line width=.5pt] (6.5,0) -- ++(0,1.5);
    \draw[pattern=north west lines, pattern color=blue, draw=none] (6.5,0) rectangle ++(0.25,1.5); 
    %филл
    \draw[draw=none] (0,0) -- ++(6.75,0);
\end{tikzpicture} 
\end{document}