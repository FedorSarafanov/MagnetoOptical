\documentclass[tikz]{standalone}

\usepackage[english,russian]{babel}
\usepackage[T2A,T1]{fontenc}
\usepackage[utf8x]{inputenc}
\usepackage
    {
        tikz,
        pgfplots,
        verbatim,
        tikz-3dplot
    }
\usetikzlibrary
    {
        arrows,
        patterns,
        angles,
        quotes,
        calc, 
        3d, 
        backgrounds, 
        positioning,
        babel
    }
\begin{document}
\begin{tikzpicture}
    %источник
    \begin{scope}[xshift=-0.5cm]
        \draw (0,0.5) rectangle node[above, yshift=1em] {1} ++(0.5,0.5);
    \end{scope}
    %поляризатор
    \begin{scope}[xshift=0.5cm]
        \draw (0,0.5) rectangle node[above, yshift=1em] {2} ++(0.5,0.5);
        \draw (0,0.5) -- ++ (0.5,0.5);
    \end{scope}
    \draw[magenta, line width=1pt] (0,0.80) -- ++ (0.5,0);  
    \draw[magenta, line width=1pt] (1,0.80) -- ++ (5.5,0);  
    \draw[blue, line width=1pt] (1,0.70) -- ++ (5.5,0);  
    %поляризация туда
    \begin{scope}[xshift=0cm]
        \xdef\angle{135}
        \draw [-,magenta, line width=1pt] (2,0.75) -- ++({180+\angle}:0.3);
        \draw [->,magenta, line width=1pt] (2,0.75) -- ++({\angle}:0.3);
    \end{scope}  
    %поляризация обратно
   \begin{scope}[xshift=0cm]
        % \draw[fill=white, dashed] (2,0.75) circle (0.25);
        \xdef\angle{45}
        \draw [-,blue, line width=1pt] (2,0.75) -- ++({180+\angle}:0.3);
        \draw [->,blue, line width=1pt] (2,0.75) -- ++({\angle}:0.3);
   
   \draw[dashed] (2,0.75) -- ++ (135:0.75);
    \draw[dashed] (1.5,0.75) coordinate (a) node[right] {}
    -- (2,0.75) coordinate (b) node[left] {}
    -- (2,0.75)++(135:1) coordinate (c) node[above right] {}
    pic["$\varphi_0$", draw=black, <-, solid, angle eccentricity=1.7, angle radius=0.4cm]
    {angle=c--b--a};

   \draw[dashed] (2,0.75) -- ++ (45:0.75);
    \draw[dashed] (1.5,0.75) coordinate (a) node[right] {}
    -- (2,0.75) coordinate (b) node[left] {}
    -- (2,0.75)++(45:1) coordinate (c) node[above right] {}
    pic["$\varphi_0+\frac{\pi}{2}$", draw=black, <-, solid, angle eccentricity=1.6, angle radius=0.47cm]
    {angle=c--b--a};
    \end{scope}      
    %поляризация до отражения
    \begin{scope}[xshift=3cm]
        % \draw[fill=white, dashed] (2,0.75) circle (0.25);
        \xdef\angle{90}
        \draw [-,magenta, line width=1pt] (2,0.75) -- ++({180+\angle}:0.3);
        \draw [->,magenta, line width=1pt] (2,0.75) -- ++({\angle}:0.3);
    \end{scope}   
    %поляризация после отражения
    \begin{scope}[xshift=3.15cm]
        % \draw[fill=white, dashed] (2,0.75) circle (0.25);
        \xdef\angle{90}
        \draw [-,blue, line width=1pt] (2,0.75) -- ++({180+\angle}:0.3);
        \draw [->,blue, line width=1pt] (2,0.75) -- ++({\angle}:0.3);

    \draw[dashed] (2,1.75) coordinate (a) node[right] {}
    -- (2,0.75) coordinate (b) node[left] {}
    -- (2,0.75)++(180:1) coordinate (c) node[above right] {}
    pic["$\varphi_0+\frac{\pi}{4}$", draw=black, <-, solid, angle eccentricity=2.1, angle radius=0.4cm]
    {angle=a--b--c};
    \end{scope}       
    %фильтр фарадея
    % \begin{scope}[xshift=3cm]
    %   \draw[fill=white] (0,0.5) rectangle node[above, yshift=1em] {3} ++(1,0.5);
    % \end{scope}  
    \begin{scope}[xshift=3cm, yshift=0.75cm]
      \draw[line width=3pt] (0,-0.25) -- ++ (1,0);
      \draw[line width=3pt] (0,0.25) -- node[above, yshift=0.2em] {3}  ++ (1,0);
      % \draw[magenta, line width=1pt] (0,0) -- (1,0);
      \draw[fill=white] (0.1,-0.1) rectangle ++ (0.8,0.2);
      % \draw[magenta!30, line width=1pt] (0.1,1pt) --++ (0.8,0);
    \end{scope}  
    %зеркало
    \draw[blue,line width=.5pt] (6.5,0) -- ++(0,1.5);
    \draw[pattern=north west lines, pattern color=blue, draw=none] (6.5,0) rectangle ++(0.25,1.5); 
    %филл
    \draw[draw=none] (0,0) -- ++(6.75,0);
\end{tikzpicture} 
\end{document}